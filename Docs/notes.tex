\documentclass[twocolumn,pra,showpacs,superscriptaddress,longbibliography]{revtex4-1}   % use preprint or twocolumn
\usepackage{graphicx}
\usepackage{amsmath}
\usepackage{amssymb}
\usepackage{epstopdf}
\usepackage{dcolumn}% Align table columns on decimal point
\usepackage{bm}% bold math
\usepackage{verbatim}
\usepackage{amsfonts}
\usepackage{cancel}
\usepackage[utf8]{inputenc}
\usepackage{graphicx}
\usepackage{setspace}
\usepackage{tabularx}
\usepackage[export]{adjustbox}



\DeclareGraphicsRule{.tif}{png}{.png}{`convert #1 `dirname #1`/`basename #1 .tif`.png}


\newcommand{\sspace}{$\enspace$}
\newcommand{\ssspace}{$\quad$}
\newcommand{\proofend}{\mbox{ }\hfill$\Box$\\}
\newcommand{\deriv}[2]{\frac{\mathrm{d} #1}{\mathrm{d} #2}}
\newcommand{\parderiv}[2]{\frac{\partial #1}{\partial #2}}
\newcommand{\ee}[1]{\cdot 10^{ #1}}
\newcommand{\bra}[1]{\left\langle #1 \right|}
\newcommand{\ket}[1]{\left| #1 \right\rangle}
\newcommand{\braopket}[3]{\left.\left\langle #1 \right|\right|#2\left|\left| #3 \right\rangle\right.}
\newcommand{\dbbraopket}[3]{\left.\left\langle#1\right.\right.\|#2\|\left.\left.#3\right\rangle\right.}
\newcommand{\braket}[2]{\left\langle #1 \right|\left. #2 \right\rangle}
\newcommand{\trace}[1]{\mathrm{Tr}\left(#1\right)}
\newcommand{\abs}[1]{\left|#1\right|}
\newcommand{\figref}[1]{Fig.~\ref{#1}}
\newcommand{\eqnref}[1]{Eqn.~\eqref{#1}}


\newcommand{\AAA}{\mathrm{\AA}}

\newcommand{\abinit}{\textit{ab initio}}
\newcommand{\abinitspace}{\textit{ab initio} }




\begin{document}

\title{Notes on calculating atom interferometer gyroscope sensitivity}
\maketitle


The goal of this project is to allow us to explore how the sensitivity of a Mach-Zehnder atom interferometer gyroscope is determined by:
\begin{itemize}
	\item choice of atom (with mass $m$ and van der Waals coefficient $C_3$)
	\item atom velocity $v$
	\item grating period $d_g$
	\item grating open fraction (defined as $w/d_g$, where $w$ is the width of the gaps between adjacent grating bars)
	\item grating longitudinal thickness $l$
	\item longitudinal distance between successive gratings $L$
\end{itemize}
Choosing the grating thickness $l$ and grating-to-grating distance $L$ is likely fairly simple: we always want smaller $l$ and larger $L$, but $l$ will probably be determined by the grating fabrication process and $L$ is limited by the maximum size of the apparatus. The grating period $d_g$ and open fraction $w/d_g$ can be optimized for a choice of atom and atom velocity $v$. Therefore, we want to be able to plot the sensitivity of a Mach-Zehnder atom interferometer gyroscope as a function of $v$ for different atoms.


Short-term gyroscope sensitivity $S$ is described as 
\begin{align}
	S = \delta\Omega \sqrt{T}
	\label{sensitivityGeneral}
\end{align}
where $\delta\Omega$ is the uncertainty in a typical measurement of rotation rate $\Omega$ and $T$ is the amount of data acquisition time required to make that measurement.
This sensitivity $S$ can be written in terms of the items listed previously.
For an atom interferometer gyroscope, the phase $\phi$ can be written as
\begin{align}
	\phi = \frac{4\pi\vec{\Omega}\cdot\vec{A}}{\lambda_{dB} v} - \phi_0
\end{align}
where $\phi_0$ is an arbitrary refrence phase, $\lambda_{dB}$ is the atoms' de Broglie wavelength, and $\vec{\Omega}\cdot\vec{A}$ is the projection of the gyroscope's rotation rate $\vec{\Omega}$ onto the plane of the interferometer with enclosed area $A$. If we let $\Omega$ be the component of $\vec{\Omega}$ normal to $\vec{A}$ and represent $\lambda_{dB}$ as $h/mv$, we can rewrite the above equation as
\begin{align}
	\delta\phi = \frac{4\pi m A}{h}\delta\Omega
\end{align}
and solve it for $\delta\Omega$ to get
\begin{align}
	\delta\Omega = \frac{h}{4\pi m A}\delta\phi
\end{align}

According to \cite{Hromada2014}






\bibliography{library}

\end{document}  