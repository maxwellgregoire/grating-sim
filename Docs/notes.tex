\documentclass[twocolumn,pra,showpacs,superscriptaddress,longbibliography]{revtex4-1}   % use preprint or twocolumn
\usepackage{graphicx}
\usepackage{amsmath}
\usepackage{amssymb}
\usepackage{epstopdf}
\usepackage{dcolumn}% Align table columns on decimal point
\usepackage{bm}% bold math
\usepackage{verbatim}
\usepackage{amsfonts}
\usepackage{cancel}
\usepackage[utf8]{inputenc}
\usepackage{graphicx}
\usepackage{setspace}
\usepackage{tabularx}
\usepackage[export]{adjustbox}



\DeclareGraphicsRule{.tif}{png}{.png}{`convert #1 `dirname #1`/`basename #1 .tif`.png}


\newcommand{\sspace}{$\enspace$}
\newcommand{\ssspace}{$\quad$}
\newcommand{\proofend}{\mbox{ }\hfill$\Box$\\}
\newcommand{\deriv}[2]{\frac{\mathrm{d} #1}{\mathrm{d} #2}}
\newcommand{\parderiv}[2]{\frac{\partial #1}{\partial #2}}
\newcommand{\ee}[1]{\cdot 10^{ #1}}
\newcommand{\bra}[1]{\left\langle #1 \right|}
\newcommand{\ket}[1]{\left| #1 \right\rangle}
\newcommand{\braopket}[3]{\left.\left\langle #1 \right|\right|#2\left|\left| #3 \right\rangle\right.}
\newcommand{\dbbraopket}[3]{\left.\left\langle#1\right.\right.\|#2\|\left.\left.#3\right\rangle\right.}
\newcommand{\braket}[2]{\left\langle #1 \right|\left. #2 \right\rangle}
\newcommand{\trace}[1]{\mathrm{Tr}\left(#1\right)}
\newcommand{\abs}[1]{\left|#1\right|}
\newcommand{\avg}[1]{\left\langle #1 \right\rangle}
\newcommand{\figref}[1]{Fig.~\ref{#1}}
\newcommand{\eqnref}[1]{Eqn.~\eqref{#1}}


\newcommand{\AAA}{\mathrm{\AA}}

\newcommand{\abinit}{\textit{ab initio}}
\newcommand{\abinitspace}{\textit{ab initio} }




\begin{document}

\title{Notes on calculating atom interferometer gyroscope sensitivity}
\maketitle


The goal of this project is to allow us to explore how the sensitivity of a Mach-Zehnder atom interferometer gyroscope is determined by:
\begin{itemize}
	\item choice of atom (and its associated van der Waals $C_3$ coefficient)
	\item atom velocity $v$
	\item grating period $d_g$
	\item grating open fraction (defined as $w/d_g$, where $w$ is the width of the gaps between adjacent grating bars)
	\item grating longitudinal thickness $l$
	\item longitudinal distance between successive gratings $L$
\end{itemize}
Choosing the grating thickness $l$ and grating-to-grating distance $L$ is likely fairly simple: we always want smaller $l$ and larger $L$, but $l$ will probably be determined by the grating fabrication process and $L$ is limited by the maximum size of the apparatus. The grating period $d_g$ and open fraction $w/d_g$ can be optimized for a choice of atom and atom velocity $v$. Therefore, we want to be able to plot the sensitivity of a Mach-Zehnder atom interferometer gyroscope as a function of $v$ for different atoms.

The short-term sensitivity $S$ of a gyroscope is described as 
\begin{align}
	S = \delta\Omega \sqrt{t}
	\label{sensitivityGeneral}
\end{align}
where $\delta\Omega$ is the uncertainty in a typical measurement of rotation rate $\Omega$ and $t$ is the amount of data acquisition time required to make that measurement. Smaller values of $S$ are preferable.
This sensitivity $S$ can be written in terms of the items listed previously.
For an atom interferometer gyroscope, the phase $\phi$ can be written as
\begin{align}
	\phi = \frac{4\pi\vec{\Omega}\cdot\vec{A}}{\lambda_{dB} v} - \phi_0
\end{align}
where $\phi_0$ is an arbitrary refrence phase, $\lambda_{dB}$ is the atoms' de Broglie wavelength, and $\vec{\Omega}\cdot\vec{A}$ is the projection of the gyroscope's rotation rate vector $\vec{\Omega}$ onto the plane of the interferometer with enclosed area $A$. If we let $\Omega$ be the component of $\vec{\Omega}$ normal to $\vec{A}$ and represent $\lambda_{dB}$ as $h/mv$, we can rewrite the above equation as
\begin{align}
	\delta\phi = \frac{4\pi m A}{h}\delta\Omega
\end{align}
and solve it for $\delta\Omega$ to get
\begin{align}
	\delta\Omega = \frac{h}{4\pi m A}\delta\phi
	\label{deltaOmega}
\end{align}

According to \cite{Lenef1997}, the uncertainty $\delta\phi$ on a single measurement of the phase of an atom interferometer is defined by
\begin{align}
	\delta\phi^2 = \frac{1}{|C|^2N}
	\label{phaseSDev}
\end{align}
where $N$ is the number of atoms detected.
Substituting Eq. \ref{phaseSDev} into Eq. \ref{deltaOmega} and the result into Eq. \ref{sensitivityGeneral}, we get 
\begin{align}
	S = \frac{h}{4\pi m A} \sqrt{\frac{t}{|C|^2N}}
	\label{sensitivity2}
\end{align}
If we take $t$ to be the unit time, we see $N/t$ is the average atom beam intensity (i.e. flux) $\avg{I}$.

The interferometer's enclosed area $A$ can be written as
\begin{align}
	A = L^2\tan{\theta_d} = L^2\tan\left(\arcsin\left(\frac{h}{mvd_g}\right)\right) \approx \frac{L^2h}{mvd_g}
\end{align}
Substituting into Eq. \ref{sensitivity2}, we get
\begin{align}
	S = \frac{vd_g}{4\pi L^2} \frac{1}{ \sqrt{|C|^2\avg{I}}}
	\label{sensitivity3}
\end{align}


\cite{Cronin2005} describes in detail how to calculate $|C|^2\avg{I}$ from the nanograting open fraction, thickness, and period. This calculation can be done with and without considering van der Waals interactions between the atoms and the grating bars. The case without van der Waals interaction represents a Mach-Zehnder atom interferometer made with Kapitza-Dirac gratings.

Note that the $\left(e_n^{Gi}\right)^2$ terms in Eq. 14 in \cite{Cronin2005} should actually be written as $\left|e_n^{Gi}\right|^2$.

van der Waals $C_3$ coefficients for alkali metals interacting with various media (including silicon nitride) are listed in atomic units in \cite{Arora2014}. To convert the $C_3$ coefficients in atomic units to SI units, multiply by 1 Hartree times the Bohr radius cubed. Useful $C_3$ coefficients are listed in Table \ref{tableC3}.

\begingroup
\begin{table}
\caption{\label{tableC3} van der Waals $C_3$ coefficients in units of $10^{-49}$ Jm$^3$ for various combinations of atoms and dielectric media \cite{Arora2014}.}
\begin{center}
\begin{tabular}{l|ll}
\hline\hline
& SiN$_x$ & SiO$_2$ \\
\hline
Li & 4.61 & 3.10 \\
Na & 5.10 & 3.49 \\
K &  7.82 & 5.42 \\
Rb & 8.85 & 6.20 \\
\hline\hline
\end{tabular}
\end{center}
\end{table}
\endgroup

\bibliography{library}

\end{document}  